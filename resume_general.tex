%% The MIT License (MIT)
%%
%% Copyright (c) 2015 Daniil Belyakov
%%
%% Permission is hereby granted, free of charge, to any person obtaining a copy
%% of this software and associated documentation files (the "Software"), to deal
%% in the Software without restriction, including without limitation the rights
%% to use, copy, modify, merge, publish, distribute, sublicense, and/or sell
%% copies of the Software, and to permit persons to whom the Software is
%% furnished to do so, subject to the following conditions:     
%%
%% The above copyright notice and this permission notice shall be included in all
%% copies or substantial portions of the Software.
%%
%% THE SOFTWARE IS PROVIDED "AS IS", WITHOUT WARRANTY OF ANY KIND, EXPRESS OR
%% IMPLIED, INCLUDING BUT NOT LIMITED TO THE WARRANTIES OF MERCHANTABILITY,
%% FITNESS FOR A PARTICULAR PURPOSE AND NONINFRINGEMENT. IN NO EVENT SHALL THE
%% AUTHORS OR COPYRIGHT HOLDERS BE LIABLE FOR ANY CLAIM, DAMAGES OR OTHER
%% LIABILITY, WHETHER IN AN ACTION OF CONTRACT, TORT OR OTHERWISE, ARISING FROM,
%% OUT OF OR IN CONNECTION WITH THE SOFTWARE OR THE USE OR OTHER DEALINGS IN THE
%% SOFTWARE.

\documentclass[]{mcdowellcv}

% For mathematical symbols
\usepackage{amsmath}
\usepackage{xcolor}
\newcommand\note[1]{\textbf{\textcolor{red}{#1}}}

% For hyperlinks
\usepackage{hyperref, color}

\definecolor{darkblue}{rgb}{0.0,0.0,0.5}
\hypersetup{
    colorlinks,
    linkcolor=darkblue,
    urlcolor=darkblue,
    anchorcolor=darkblue,
    citecolor=darkblue
}

% Set applicant's personal data for header
\name{Pierre Beckmann}
\address{+41 78 682 82 97\linebreak pierrebeckmann@gmail.com}
\contacts{\href{https://orcid.org/0000-0001-9247-4841}
             {Orcid}
 \linebreak
\href{https://scholar.google.com/citations?user=N8AW7IgAAAAJ}{Google Scholar}}

         

%\renewcommand{\baselinestretch}{1.05} 

\begin{document}
    \makeheader
    

 %\begin{cvsection}{About}

%\begin{cvsubsection}{}{}{}
%I'm a researcher with a dual education in Artificial Intelligence and in Philosophy. I'm currently looking for a PhD position.
%\end{cvsubsection}
% \end{cvsection}

\begin{cvsection}{Education}
\begin{cvsubsection}{EPFL Lausanne - IDIAP, PhD Student}{}{May\ 2025 -- Now}
            \begin{itemize}
                \item PhD Student in the Neuro-Symbolic AI group of Prof. Andre Freitas, working for the \href{https://data.snf.ch/grants/grant/10004303}{M-RATIONAL} project.
            \end{itemize}
\end{cvsubsection}
\begin{cvsubsection}{UNIL Lausanne, M.A. Philosophy}{}{Sept.\ 2020 -- June 2023}
            \begin{itemize}
                \item With 40 credits to catch up. Focus on Philosophy of Science, Kantian Philosophy and Phenomenology.  
                \item Title of master thesis: ``Le transcendantal, le physiologique et le computationnel: le predictive processing à partir de Kant et de Helmholtz''.
            \end{itemize}
\end{cvsubsection}
\begin{cvsubsection}{Humboldt University Berlin, Exchange Semester in Philosophy}{}{April\ 2022 -- July 2022}
            \begin{itemize}
                \item Exchange Semester in Philosophy at Humboldt University Berlin.
            \end{itemize}
\end{cvsubsection}
        \begin{cvsubsection}{UNIL Lausanne, Propedeutical Year, Philosophy, Sociology and French}{}{Sept.\ 2019 -- Aug.\ 2020}
            \begin{itemize}
                \item Propedeutic year in Philosophy, Sociology and Modern French. 
            \end{itemize}
        \end{cvsubsection}
        \begin{cvsubsection}{ETH Zurich, M.Sc. Computational Sciences and Engineering}{}{Sept.\ 2016 -- Jan. 2019}
            \begin{itemize}
                \item Research-centered program with a focus on Machine Learning, NLP, Computer Vision and HPC.
                \item Title of master thesis: ``Story understanding and story generation with deep learning".
            \end{itemize}
        \end{cvsubsection}
        \begin{cvsubsection}{EPFL Lausanne, B.Sc. Life Sciences and Technologies}{}{Sept.\ 2013 -- July 2016}
            \begin{itemize}
               \item Engineering program with Mathematics, Physics and Computer Science as well as Biology and Chemistry.
            \end{itemize}
        \end{cvsubsection}
    \end{cvsection}

   
    \begin{cvsection}{Research experience}
        \begin{cvsubsection}{University of Bern, Research Assistant, Philosophy of AI}{}{Mar. 2023 -- Dec. 2024}
                \begin{itemize}
                  \item Worked on personal research in the field of philosophy of deep learning under the supervision of Prof. Claus Beisbart.
                \end{itemize}
            \end{cvsubsection}
        \begin{cvsubsection}{UNIL BCUL, Computer Science Engineer, 60\%}{}{Oct. 2022 -- Feb. 2023}
            \begin{itemize}
              \item Designed and implemented software to deal with UNIL's digital libraries; collaborated with EPFL on project Impresso.
            \end{itemize}
        \end{cvsubsection}
 
	\begin{cvsubsection}{EPFL LSIR, Research Engineer, 50\%}{}{Sept. 2019 -- Aug. 2021}
            \begin{itemize}
              \item Worked in a research team to implement and deploy multimodal-LLM-based video retrieval systems.
            \end{itemize}
        \end{cvsubsection}
        \begin{cvsubsection}{Logitech Lausanne, Scientific advisor, 10\%}{}{Sept. 2019 -- April 2023}
            \begin{itemize}
              \item Consulting for internship and master thesis projects in the CTO-AI department in deep learning applied audio.
            \end{itemize}
        \end{cvsubsection}
	\begin{cvsubsection}{Logitech Lausanne, Research Engineering Intern}{}{Feb. 2019 -- July 2019}
            \begin{itemize}
              \item Researched product-feasible deep learning solutions for audio-related tasks.
            \end{itemize}
        \end{cvsubsection}
        \begin{cvsubsection}{Disney Research Zurich (DRZ), Master Thesis and Semester Project}{}{March 2018 -- Feb. 2019}
            \begin{itemize}
              \item Conducted research on story understanding and generation using deep learning approaches.
            \end{itemize}
        \end{cvsubsection}
        \begin{cvsubsection}{ETHZ Computer Vision and Geometry Group (CVG), Research Assistant, 10\%}{}{Oct. 2017 -- Sept 2018}
            \begin{itemize}
                \item Developed a mixed reality application for the Hololens from an idea to a working first version.
            \end{itemize}
        \end{cvsubsection}

    \end{cvsection}  

    \begin{cvsection}{Teaching experience}

      \begin{cvsubsection}{University of Bern, Assistant lecturer, Philosophy of AI}{}{Oct. 2024 -- Dec. 2024}
\begin{itemize}
 \item Co-teaching two CAS courses of Advanced Machine Learning, delivering one third of the lectures:
 {\setlength\itemindent{25pt} \item Philosophy and Ethics of Extended Cognition and Artificial Intelligence.}
 {\setlength\itemindent{25pt} \item NLP: Philosophical and Ethical Aspects.}
\end{itemize}
\end{cvsubsection}
       
  \begin{cvsubsection}{UNIL, Tutor, Philosophy of the Mind}{}{Feb. 2023 -- Jun. 2023}
\begin{itemize}
\item Tutor for an introductory course to contemporary philosophy of mind, supporting first-year university students.
\end{itemize}
\end{cvsubsection}

  \begin{cvsubsection}{UNIL, Tutor, General and Systematic Philosophy}{}{Sept. 2021 -- Jan. 2022}
\begin{itemize}
\item Tutor for an introductory course to phenomenology, supporting first-year university students.
\item Designed and organized three philosophy methodology workshops.
\end{itemize}
\end{cvsubsection}

%  \begin{cvsubsection}{Ecole Lemania, School teacher, Computer science}{}{Sept. 2022 -- Jun. 2023}
%\begin{itemize}
%\item Teacher for students aged 13-14. 
%\item Designed the annual curriculum from scratch, focused on teaching Python.
%\end{itemize}
%\end{cvsubsection}

\begin{cvsubsection}{EPFL, Student Assistant, Chemistry}{}{Sept. 2014 -- July 2016} \begin{itemize} \item Student assistant in chemistry (twice) and in organic chemistry II for EPFL students. \end{itemize} \end{cvsubsection}


    \end{cvsection}    


\begin{cvsection}{Publications}
    \begin{cvsubsection}{}{}{}
        
        \textbf{Publications in Philosophy:} \vspace{0.1cm}
        \begin{itemize}
            \item \underline{Pierre Beckmann}, Matthieu Queloz, "Mechanistic Indicators of Understanding in Large Language Models", \textit{arxiv} 2025 [\href{https://arxiv.org/abs/2507.08017}{Paper}]
        \end{itemize}
        \begin{itemize}
            \item \underline{Pierre Beckmann}, "New Horizons in Machine Understanding: Explanatory and Objectual Understanding in Deep Learning Video Generation Models", \textit{To appear in Synthese} 2025
        \end{itemize}
        \begin{itemize}
            \item \underline{Pierre Beckmann}, Guillaume Köstner, Ines Hipolito, "An alternative to cognitivism: computational phenomenology and deep learning", Minds and Machines 2023 [\href{https://link.springer.com/article/10.1007/s11023-023-09638-w}{Paper}]
        \end{itemize}
        

        \vspace{0.1cm}
        \textbf{Publications in Deep Learning:} \vspace{0.1cm}
        \begin{itemize}
        \setlength\itemsep{0.4em}
            \item \underline{Pierre Beckmann*}, Mikolaj Kegler*, Milos Cernak, "Deep speech inpainting of time-frequency masks", Interspeech 2020 [\href{https://arxiv.org/abs/1910.09058}{Paper}|\href{https://mkegler.github.io/SpeechInpainting/}{Demo}]
            \item \underline{Pierre Beckmann*}, Mikolaj Kegler*, Milos Cernak, "Word-Level Embeddings for Cross-Task Transfer Learning in Speech Processing", EUSIPCO 2021 [\href{https://arxiv.org/abs/1910.09909}{Paper}|\href{https://github.com/bepierre/SpeechVGG}{Code}]
            \item Neil Scheidwasser-Clow, Mikolaj Kegler, \underline{Pierre Beckmann}, Milos Cernak, "SERAB: A multi-lingual benchmark for speech emotion recognition", ICASSP 2021 [\href{https://arxiv.org/abs/2110.03414}{Paper}|\href{https://github.com/neclow/serab}{Code}]
            \item Gasser Elbanna, Alice Biryukov, Neil Scheidwasser-Clow, Lara Orlandic, Pablo Mainar, Mikolaj Kegler, \underline{Pierre Beckmann}, Milos Cernak, "Hybrid Handcrafted and Learnable Audio Representation for Analysis of Speech Under Cognitive and Physical Load", Interspeech 2022 [\href{https://arxiv.org/abs/2203.16637}{Paper}|\href{https://github.com/GasserElbanna/serab-byols}{Code}]
            \item Gasser Elbanna, Neil Scheidwasser-Clow, Mikolaj Kegler, \underline{Pierre Beckmann}, Karl El Hajal, Milos Cernak, "Byol-s: Learning self-supervised speech representations by bootstrapping", 2023, PMLR, 3rd place in \href{https://hearbenchmark.com/hear-leaderboard.html}{HEAR benchmark} [\href{https://arxiv.org/abs/2206.12038}{Paper}|\href{https://github.com/GasserElbanna/serab-byols}{Code}]
        \end{itemize}

        \vspace{0.1cm}
        \textbf{Contributions to Course Books:} \vspace{0.1cm}
        \begin{itemize}
            \item Contributed to three chapters of the course book \href{https://github.com/simbrain/NeuralNetworksCogSciBook}{\textit{Neural Networks for Cognitive Sciences}}, co-written with Prof. Jeff Yoshimi from University of California Merced, 2024:
            \begin{itemize}
                \item History of Neural Networks
                \item Convolutional Neural Networks
                \item Transformer Architecture and LLMs
            \end{itemize}
        \end{itemize}

        \vspace{0.1cm}
        \textbf{Peer Reviews:} \vspace{0.1cm}
        \begin{itemize}
            \item Ubiquity Press, 2025
            \item Knowledge in Society, Springer Nature, 2023.
        \end{itemize}

    \end{cvsubsection}
\end{cvsection}



 \begin{cvsection}{Talks}

\begin{cvsubsection}{}{}{}
\begin{itemize}
\setlength\itemsep{0.4em}

\item February, 2025. \textit{Computational phenomenology and deep learning}. \href{https://ideas-ncbr.pl/en/events/calculating-experience-phenomenology-ai-for-mental-health-seminar/}{\textit{Calculating Experience?} Seminar}, St Catherine's College, Oxford, UK.

\item November, 2024. \textit{Les neurosciences computationnelles à l'épreuve de la méthode transcendantale}. Conference series for the 300 years of Kant, UNIL, Lausanne, Switzerland.

\item October, 2024. \textit{Could deep learning video generation models understand the physical world? A philosophical perspective.} Talk at IDIAP, EPFL, Switzerland.

\item November, 2022. \textit{Computational phenomenology and deep learning}. Conference series on AI and consciousness, UNIL, Switzerland.

\end{itemize}
\end{cvsubsection}

 
 \end{cvsection}

    \begin{cvsection}{Skills and other employments}
        \begin{cvsubsection}{}{}{}
            \begin{itemize}
            \setlength\itemsep{0.4em}
            \item \textbf{Languages:} French (Native), German (Native), English (Proficient).
               \item \textbf{Programming skills:} \\
                    -- Programming languages: Python, C++, C, C\#, Matlab, R, Bash/UNIX, ROS. \\
                 -- Deep learning: Tensorflow, PyTorch, Keras, Mlflow. \\
		 -- High Performance Computing: OpenMP, MPI, Vectorization, CUDA. 
               
   
            \item \textbf{Employments in scientific mediation}: \\ -- Scientific mediator at EPFL (2021-2023). \\ -- Scientific reporter at simplyscience.ch (2021-2022). 
            \item \textbf{Associations}:\\ -- Committee member of the \href{https://philo-vaud.ch/}{Groupe Vaudois de Philosophie} (2024-Now). \\ -- Vice-president of ORPHI, the association of philosophy students of UNIL (2022-2023).
             \item \textbf{Grants}: Seed money, University of Bern, 10'000 CHF. Funded research period from Jul. 2023 to Sept. 2023.
              
               \end{itemize}

        \end{cvsubsection}
        
    \end{cvsection}

\end{document}
